%------------------Settings-------------------------

\documentclass[12pt]{article}
\usepackage[utf8]{inputenc}
\usepackage[russian]{babel}
\usepackage{amsmath,amssymb}
\usepackage{graphics}
\usepackage{pbox}
\usepackage[x11names]{xcolor}
\definecolor{brightmaroon}{rgb}{0.76, 0.13, 0.28}
\definecolor{royalazure}{rgb}{0.0, 0.22, 0.66}
\usepackage[colorlinks=true,linkcolor=royalazure]{hyperref}
\usepackage{tikz, tkz-fct, pgfplots}
\usetikzlibrary{arrows}
\usepackage{geometry}
\geometry{
	a4paper,
	total={170mm,257mm},
	left=20mm,
	top=20mm
} 
\usepackage[labelsep=period]{caption}

% ----------------- Commands ----------------- 

\newcommand{\eps}{\varepsilon}
\newcommand\tline[2]{$\underset{\text{#1}}{\text{\underline{\hspace{#2}}}}$}

% ----------------- Set graphics path ----------------- 
\graphicspath{{img/}}
\begin{document}
\pagestyle{empty}

% ----------------------Title----------------------------------
\centerline{\large Министерство науки и высшего образования}	
\centerline{\large Федеральное государственное бюджетное образовательное}
\centerline{\large учреждение высшего образования}
\centerline{\large ``Московский государственный технический университет}
\centerline{\large имени Н.Э. Баумана}
\centerline{\large (национальный исследовательский университет)''}
\centerline{\large (МГТУ им. Н.Э. Баумана)}
\hrule
\vspace{0.5cm}
\begin{figure}[h]
\center
\includegraphics[height=0.35\linewidth]{bmstu-logo-small.png}
\end{figure}
\begin{center}
	\large	
	\begin{tabular}{c}
		Факультет ``Фундаментальные науки'' \\
		Кафедра ``Высшая математика''		
	\end{tabular}
\end{center}
\vspace{0.5cm}
\begin{center}
	\LARGE \bf	
	\begin{tabular}{c}
		\textsc{Отчёт} \\
		по учебной практике \\
		за 1 семестр 2020---2021 гг.
	\end{tabular}
\end{center}
\vspace{0.5cm}
\begin{center}
	\large
	\begin{tabular}{p{5.3cm}ll}
		\pbox{5.45cm}{
			Руководитель практики,\\
			ст. преп. кафедры ФН1} 	& \tline{\it(подпись)}{5cm} & Кравченко О.В. \\[0.5cm]
		студент группы ФН1--11 		& \tline{\it(подпись)}{5cm} & Францева Е.М.
	\end{tabular}
\end{center}
\vfill
\begin{center}
	\large	
	\begin{tabular}{c}
		Москва, \\
		2020 г.
	\end{tabular}
\end{center}
 \newpage
 \newpage	
\tableofcontents
%------------------Table of contents----------------------
\newpage
\section{Цели и задачи практики}	
\subsection{Цели}
--- развитие компетенций, способствующих успешному освоению материала бакалавриата и необходимых в будущей профессиональной деятельности.
\subsection{Задачи}
\begin{enumerate}
\item Знакомство с программными средствами, необходимыми в будущей профессиональной деятельности.
\item Развитие умения поиска необходимой информации в специальной литературе и других источниках.
\item Развитие навыков составления отчётов и презентации результатов.
\end{enumerate}
\subsection{Индивидуальное задание}	
\begin{enumerate}
\item Изучить способы отображения математической информации в системе вёртски \LaTeX.
\item Изучить возможности  системы контроля версий \textsf{Git}.
\item Научиться верстать математические тексты, содержащие формулы и графики в системе \LaTeX.
Для этого, выполнить установку свободно распространяемого дистрибутива \textsf{TeXLive} и оболочки \textsf{TeXStudio}.
\item Оформить в системе \LaTeX типовые расчёты по курсу математического анализа согласно своему варианту.
\item Создать аккаунт на онлайн ресурсе \textsf{GitHub} и загрузить исходные \textsf{tex}--файлы 
и результат компиляции в формате \textsf{pdf}.
\end{enumerate} 
%---------------------------------------------------------------
\newpage
\section{Отчёт}
Актуальность темы продиктована необходимостью владеть системой вёрстки \LaTeX и средой вёрстки \textsf{TeXStudio} для
отображения текста, формул и графиков. Полученные в ходе практики навыки могут быть применены при написании
курсовых проектов и дипломной работы, а также в дальнейшей профессиональной деятельности.
Ситема вёрстки \LaTeX содержит большое количество инструментов (пакетов), упрощающих отображение информации в различных 
сферах инженерной и научной деятельности. 
%-----------------------------------------------------------------
\newpage
\section{Индивидуальное задание}
%\subsection{Элементарные функции и их графики.}
%\input{src/part1.tex}
%==============================================================================
\subsection{Пределы и непрерывность.}
%---------------------------- Problem 1----------------------------------
\subsubsection*{\center Задача № 1.}
{\bf Условие.~}
Дана последовательность $a_{n}=\dfrac{1-2n^2}{2+4n^2}$ и число $c=\dfrac{-1}{2}$. Доказать, что $\lim\limits_{x\rightarrow\infty} a_{n}=c $, а именно, для каждого $\varepsilon>0$ найти наименьшее натуральное число  $N{=}N(\varepsilon)$ такое, что $|a_{n}-c|<\varepsilon$ для всех $n>N(\varepsilon)$. Заполнить таблицу: 
\begin{center}
\begin{tabular}{ | p{25pt} | c | c | c | c |}
\hline
$\varepsilon$& $0{,}1$ & $0{,}01$ & $0{,}001$ \\ \hline
$N(\varepsilon)$ &   &   &\\
\hline
\end{tabular}
\end{center}
\medskip
%=====================================================================
{\bf Решение.~}
Рассмотрим неравенство $a_{n}-c<\varepsilon$, $\forall\varepsilon>0$, учитывая выражение для $a_{n}$ и $c$ из условия варианта, получим 
$$\left|\frac{1-2n^2}{2+4n^2}+\frac{1}{2}\right|<\varepsilon$$
Неравенство запишем в виде двойного неравенства и приведём выражение под знаком модуля к общему знаменателю, получим
$${-}\varepsilon <\dfrac{1}{1+2n^2}<\varepsilon$$
Заметим, что левое неравенство выполнено для любого номера $n\in \mathbb{N}$ поэтому, будем рассматривать правое неравенство
$$\frac{1}{1+2n^2}<\varepsilon$$
Выполнив цепочку преобразований, перепишем неравенство относительно $n$, и, учитывая, что $n\in \mathbb{N}$, получим 
$$\dfrac{1}{1+2n^2)}<\varepsilon,$$
$$1+2n^2>\dfrac{1}{\varepsilon},$$
$$n^2>\dfrac{1}{2\varepsilon}-\dfrac{1}{2},$$
$$n>\sqrt{\dfrac{1}{2\varepsilon}-\dfrac{1}{2}},$$
$$N(\varepsilon)=\biggl[\sqrt{\dfrac{1}{2\varepsilon}-\dfrac{1}{2}}\biggr],$$
где $[\;]$ -- целая часть от числа. Заполним таблицу:
\begin{center}
\begin{tabular}{ | p{25pt} | c | c | c | c |}
\hline
$\varepsilon$& $0{,}1$ & $0{,}01$ & $0{,}001$ \\ \hline
$N(\varepsilon)$ & 2  & 7 & 22\\
\hline
\end{tabular}
\end{center}
{\bf Проверка:~}
$$|a_{3}-c|=\dfrac{1}{19}<0{,}1,$$
$$|a_{8}-c|=\dfrac{1}{129}<0{,}01,$$
$$|a_{23}-c|=\dfrac{1}{1059}<0{,}001.$$
\newpage
% ---------------------------- Problem 2----------------------------------
\subsubsection*{\center Задача № 2.}
{\bf Условие.~}
Вычислить пределы функций
$$
\begin{array}{cc}
\text{\bf(а):} &  \lim\limits_{x\rightarrow-1}\dfrac{x^3-2x-1}{x^5-2x-1} , \\[10pt]
\text{\bf(б):} & \lim\limits_{x\rightarrow+\infty} \dfrac{3x^2-x\sqrt[4]{5+4x^2}+3x}{1+5x^2+x} ,\\[10pt]
\text{\bf(в):} & \lim\limits_{x\rightarrow+3} \dfrac{\sqrt[3]{9x}-3}{\sqrt{3+x}-\sqrt{2x}},\\[10pt]
\text{\bf(г):} & \lim\limits_{x\rightarrow0} (\cos{\pi x})^{\frac{1}{x\sin{x}}}, \\[10pt]
\text{\bf(д):} & \lim\limits_{x\rightarrow0} \left(\dfrac{e^{3x}-e{x}}{x}\right)^{\log_2 \cos{(x+\frac{\pi}{4})}} , \\[10pt]
\text{\bf(е):}  & \lim\limits_{x\rightarrow+1} \dfrac{2-\sqrt{3x+1}}{\cos{\dfrac{\pi x}{2}}} . \\
\end{array}
$$
\\
{\bf Решение.~}\\
\\
\text{\bf(а):}
$$
\begin{array}{l}
\lim\limits_{x\rightarrow-1} \dfrac{x^3-2x-1}{x^5-2x-1} =\left[\dfrac{0}{0} \right]= \lim\limits_{x\rightarrow-1}  \dfrac{(x+1)(x^2-x-1)}{(x+1)(x^4-x^3+x^2-x-1)} = \lim\limits_{x\rightarrow-1}  \dfrac{x^2-x-1}{x^4-x^3+x^2-x-1}=\\ \medskip{}{}=\dfrac{1+1-1}{1+1+1+1-1}=\dfrac{1}{3}
\end{array}
$$
\\
\text{\bf(б):}
$$
\begin{array}{l}
\lim\limits_{x\rightarrow+\infty} \dfrac{3x^2-x\sqrt[4]{5+4x^2}+3x}{1+5x^2+x}=\lim\limits_{x\rightarrow+\infty} \dfrac{x^2}{x^2}\dfrac{3-\sqrt[4]{\dfrac{1}{x^4}+\dfrac{4}{x^2}}+\dfrac{3}{x}}{\dfrac{1}{x^2}+5+\dfrac{1}{x}}=\dfrac{3}{5}
\end{array}
$$
\text{\bf(в):}
 $$
 \begin{array}{l} 
 \lim\limits_{x\rightarrow+3} \dfrac{\sqrt[3]{9x}-3}{\sqrt{3+x}-\sqrt{2x}} = \left[\dfrac{0}{0} \right]= \lim\limits_{x\rightarrow+3} \dfrac{(9x-27)(\sqrt{3+x}+\sqrt{2x})}{(3+x-2x)(\sqrt[3]{81x^2}+3\sqrt[3]{9x}+9)}=\\ \medskip{}{} = \lim\limits_{x\rightarrow+3} \dfrac{9(x-3)(\sqrt{3+3}+\sqrt{6})}{(3-x)\sqrt[3]{129}+3\sqrt[3]{27}+9}=\dfrac{-9*2\sqrt{6}}{27}=-\dfrac{2\sqrt{6}}{3}
 \end{array}
 $$
 \\
\text{\bf(г):}
$$
\begin{array}{l}
\lim\limits_{x\rightarrow0} (\cos{\pi x})^{\frac{1}{x\sin{x}}}=\left[1^{\infty}\right]=\lim\limits_{x\rightarrow0} (1+(\cos(\pi x)-1))^{\frac{1}{\cos(\pi x)-1}*\frac{\cos(\pi x )-1}{x\sin(\pi x)}}=\lim\limits_{x\rightarrow0} e^{\frac{\cos(\pi x)-1}{x\sin(x)}}=\\ \medskip{}{}= \left|\sin(\pi x)\sim \pi x, \cos(\pi x) \sim 1-\frac{(\pi x)^2}{2} \right|=\lim\limits_{x\rightarrow0} e^{\frac{1-\frac{\pi^2 x^2}{2}-1}{x \pi x}}=\lim\limits_{x\rightarrow0} e^{\frac{- \pi^2 x^2}{2\pi x^2}}=e^{-\frac{\pi}{2}}
\end{array}
$$
\\
\text{\bf(д):}
$$
\begin{array}{l}
 \lim\limits_{x\rightarrow0} \left(\dfrac{e^{3x}-e^x}{x}\right)^{\log_2 \cos{(x+\frac{\pi}{4})}}= \lim\limits_{x\rightarrow0} \left(\dfrac{e^{x}(e^{2x}-1)}{x}\right)^{\log_2 \cos{(x+\frac{\pi}{4})}}= \lim\limits_{x\rightarrow0} \left(\dfrac{e^x*2x}{x}\right)^{\log_2 \cos{(x+\frac{\pi}{4})}}=\\ \medskip{}{}=2^{log_2 {\frac{\sqrt{2}}{2}}}={\dfrac{\sqrt{2}}{2}}
\end{array}
$$
\text{\bf(е):}
$$
\begin{array}{l}
\lim\limits_{x\rightarrow1} \dfrac{2-\sqrt{3x+1}}{\cos{\dfrac{\pi x}{2}}}=\left[\dfrac{0}{0} \right]=\left| y=x+1, y \to 0\right|=\lim\limits_{y\rightarrow0} \dfrac{2-\sqrt{3(y+1)+1}}{\cos{\dfrac{\pi (y+1)}{2}}}=\lim\limits_{y\rightarrow0} \dfrac{\sqrt{3y+4}-2}{\sin{\dfrac{\pi y}{2}}}=\\ \medskip{}{}=\left|\sin{\dfrac{\pi y}{2}} \sim \dfrac{\pi y}{2}\right|=\lim\limits_{y\rightarrow0} \dfrac{2(\sqrt{\frac{3}{4}y+1}-1)}{\dfrac{\pi y}{2}}=\dfrac{4}{\pi}\lim\limits_{y\rightarrow0} \dfrac{\frac{1}{2}*\frac{3}{4}y}{y}=\dfrac{4}{\pi}*\dfrac{3}{4}*\dfrac{1}{2}=\dfrac{3}{2 \pi}
\end{array}
$$
% ---------------------------- Problem 3----------------------------------
\subsubsection*{\center Задача № 3.}
{\bf Условие.~}\\
\text{\bf(а):} Показать, что данные функции
$f(x)$ и $g(x)$ являются бесконечно малыми или бесконечно большими
при указанном стремлении аргумента. \\
\text{\bf(б):} Для каждой функции $f(x)$ и $g(x)$ записать главную часть
(эквивалентную ей функцию)  вида $C(x-x_0)^{\alpha}$ при $x\rightarrow x_0$ или $Cx^{\alpha}$
при $x\rightarrow\infty$, указать их порядки малости (роста). \\
\text{\bf(в):} Сравнить функции $f(x)$ и $g(x)$ при указанном стремлении.
\begin{center}
	\begin{tabular}{|c|c|c|}
		\hline
		№ варианта & функции $f(x)$ и $g(x)$ & стремление \\[6pt]
		\hline
		23 & $f(x) = \dfrac{x\arctan x}{\sqrt{4x+3}},~g(x)=\sqrt{x}-\sqrt[3]{x}$ & $x\rightarrow+\infty$ \\
		\hline
		\end{tabular}
		\bigskip
		\\
		{\bf Решение.~}\\
		\end{center}
		\medskip
		\text{\bf(а):}~Покажем, что $f(x)$ и $g(x)$ бесконечно большие функции,
$$
\begin{array}{l} 
\lim\limits_{x\rightarrow+ \infty} f(x)=\lim\limits_{x\rightarrow +\infty} \dfrac{x\arctan x}{\sqrt{4x+3}}=\left|\arctan x \sim \dfrac{\pi}{2} \right|=\lim\limits_{x\rightarrow+\infty} \dfrac{\frac{\pi x}{2}}{\sqrt{4x+3}}=\lim\limits_{x\rightarrow+\infty} \dfrac{\pi \sqrt{x}}{2\sqrt{4+\frac{3}{x}}}=\lim\limits_{x\rightarrow+\infty} \dfrac{\pi \sqrt{x}}{4}=\infty , \\
 
\lim\limits_{x\rightarrow+ \infty} g(x)= \lim\limits_{x\rightarrow \infty}  \sqrt{x}-\sqrt[3]{x}=\lim\limits_{x\rightarrow +\infty} \sqrt{x}\biggl(1-\dfrac{1}{\sqrt[6]{x}}\biggr)=\lim\limits_{x\rightarrow +\infty} \sqrt{x}=\infty.
\end{array}
$$
\text{\bf(б):}~Так как $f(x)$ и $g(x)$ бесконечно большие функции, то эквивалентными им будут функции вида 
$Cx^{\alpha}$ при $x\rightarrow+\infty$. Найдём эквивалентную для $f(x)$ из условия
$$
\lim\limits_{x\rightarrow+\infty}\dfrac{f(x)}{x^{\alpha}} = C,
$$
где $C$ --- некоторая константа. Рассмотрим предел
$$
\lim\limits_{x\rightarrow +\infty} \dfrac{x\arctan x}{x^{\alpha}\sqrt{4x+3}}=\lim\limits_{x\rightarrow+\infty} \dfrac{\pi \sqrt{x}}{4x^{\alpha}}
$$
При $\alpha=\dfrac{1}{2}$ последний предел равен $\dfrac{\pi}{4}$, отсюда $C=\dfrac{\pi}{4}$ и 
$$
f(x)\sim \dfrac{\pi}{4} \sqrt{x}~\text{при}~x\rightarrow+\infty.
$$
Аналогично, рассмотрим предел
$$
\lim\limits_{x\rightarrow+\infty}\dfrac{g(x)}{x^{\alpha}}=\lim\limits_{x\rightarrow \infty} \dfrac{\sqrt{x}-\sqrt[3]{x}}{x^{\alpha}} =\lim\limits_{x\rightarrow +\infty} \dfrac{\sqrt{x}}{x^{\alpha}}
$$
При $\alpha=\dfrac{1}{2}$ последний предел равен $1$, отсюда $C=1$ и
$$
g(x)\sim\sqrt{x}~\text{при}~x\rightarrow+\infty.
$$
\text{\bf(в):}~Для сравнения функций $f(x)$ и $g(x)$ рассмотрим предел их отношения при указанном стремлении
$$
\lim\limits_{x\rightarrow\infty}\dfrac{f(x)}{g(x)}.
$$
Применим эквивалентности, определенные в пункте (б), получим
$$
\lim\limits_{x\rightarrow+\infty}\dfrac{f(x)}{g(x)} = \lim\limits_{x\rightarrow+\infty}\dfrac{\dfrac{\pi}{4}\sqrt{x}}{\sqrt{x}}=\dfrac{\pi}{4}
$$
Отсюда, $f(x)$ есть бесконечно большая функция более высокого порядка роста, чем $g(x)$.
%=================================================================================================================================
%\subsection{Приложения дифференциального исчисления.}
%\input{src/part3.tex}
\newpage
\addcontentsline{toc}{section}{Список литературы}
\begin{thebibliography}{99}
\bibitem{book01} Львовский С.М. Набор и вёрстка в системе \LaTeX, 2003 c.
\bibitem{book02} Котельников И.А., Чеботаев П.3. \LaTeX по-русски.
\end{thebibliography}
\end{document}